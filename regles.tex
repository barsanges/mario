\documentclass[a4paper,11pt,headings=normal]{scrartcl}

\usepackage[utf8]{inputenc}
\usepackage[T1]{fontenc}
\usepackage[french]{babel}
\usepackage{biblatex}
\usepackage{amsthm}
\usepackage{newtx, helvet, courier} % Fontes
\usepackage{booktabs, multirow, hyperref, xcolor}
\usepackage{enumitem}
\usepackage{tikz}
\usetikzlibrary{calc}
\frenchsetup{AutoSpaceFootnotes=false}

\newtheoremstyle{mythm}% name
{0.5\baselineskip plus 0.2\baselineskip minus 0.2\baselineskip}% Space above
{0.5\baselineskip plus 0.2\baselineskip minus 0.2\baselineskip}% Space below
{\itshape}% Body font
{\parindent}% Indent amount
{}% Theorem head font
{.}% Punctuation after theorem head
{.5em}% Space after theorem head
{\thmname{{\scshape#1}}\thmnumber{ #2}\thmnote{~---~#3}}% Theorem head spec (can be left empty, meaning ‘normal’)
\theoremstyle{mythm}
\newtheorem{cas}{Cas}

\definecolor{rouge}{RGB}{191, 35, 41}

\renewcommand{\sectionformat}{\thesection\autodot\enskip}
\renewcommand{\subsectionformat}{\thesubsection\autodot\enskip}
\RedeclareSectionCommand[indent=\the\parindent]{subsection}
\setkomafont{disposition}{}
\setkomafont{section}{\bfseries\large}
\setkomafont{subsection}{\itshape\normalsize}
\setkomafont{paragraph}{\itshape}

\setlist[1]{noitemsep,topsep=0mm,left=\parindent}
\setlist[2]{noitemsep,topsep=0mm,left=0pt}
\setlist[enumerate,1]{label=(\alph*)}

\urlstyle{same}

\addbibresource{citations.bib}

\hypersetup{
    colorlinks=true,
    linkcolor=black,
    citecolor=black,
    urlcolor=blue,
    pdftitle={Mario - Règles},
}

\begin{document}

\begin{flushleft}
{\small \sffamily \url{https://github.com/barsanges/mario}}\vspace{0.8em}

{\raggedleft \bfseries\LARGE Règles}\vspace{0.8em}

{\itshape Version \input{VERSION}}
\end{flushleft}

\section{Principes}

\emph{Mario} est un jeu d'affrontement direct en temps réel qui oppose
deux joueurs sur un plateau de jeu en forme de graphe. L'objectif est
d'éliminer complétement l'adversaire du plateau. Pour cela, les
joueurs bénéficient chacun d'un afflux continu de troupes, qui leur
permettent de prendre progressivement le contrôle de portions du
graphe et de détruire des troupes adverses.

\section{Plateau de jeu}

Le plateau est constitué d'un graphe connexe, décomposé en $n$ nœuds
et $p$ arêtes. Chaque arête relie exactement deux nœuds distincts. En
début de partie, certains nœuds sont des \og centres \fg{} qui
génèrent des troupes en continu pour un des deux camps. On note :
\begin{itemize}
  \item $P_i \subset \llbracket 1 ; p \rrbracket$ l'ensemble des
    arêtes dont le nœud $i \in \llbracket 1 ; n \rrbracket$ constitue
    une des extrêmités ;

  \item $\alpha_{q,i} \geq 0$ la quantité instantanée de troupes
    générée au nœud $i \in \llbracket 1 ; n \rrbracket$ pour le camp
    $q \in \{1, 2\}$ tant qu'il n'a pas perdu le contrôle de ce nœud ;

  \item $C_q = \{ i \in \llbracket 1 ; n \rrbracket / \alpha_{q,i} > 0
    \}$, l'ensemble des centres du camp $q \in \{1, 2\}$ au début de
    la partie ; le plateau de jeu est tel que $C_q \cap C_{\bar q} =
    \emptyset$\footnote{Pour $q \in \{1, 2\}$, on note par convention
    ${\bar q} = 2$ si $q = 1$, et ${\bar q} = 1$ si $q = 2$.} ;

  \item $g_k \in \llbracket 1 ; n \rrbracket$ et $d_k \in \llbracket 1
    ; n \rrbracket$ les deux extrêmités de l'arête $k$, avec la
    convention $g_k < d_k$ ;

  \item $\lambda_k > 0$ la longueur de l'arête $k \in \llbracket 1 ; p
    \rrbracket$ ;

  \item $\gamma_k > 0$ la section de l'arête $k \in \llbracket 1 ; p
    \rrbracket$ ;

  \item $\phi_k > 0$ la quantité maximale de troupes qu'il est
    possible d'injecter ou de soutirer sur l'arête $k \in \llbracket 1
    ; p \rrbracket$ à un instant donné via une de ses extrêmités
    ($g_k$ ou $d_k$).
\end{itemize}

\section{Déroulé}

Dans chaque camp, de nouvelles troupes arrivent en continu dans les
centres du joueur et poussent les troupes déjà présentes sur le
plateau. Celles-ci se répandent sur les arêtes libres, jusqu'à ce
qu'elles rencontrent des troupes adverses, ce qui entraîne des
destructions de part et d'autre.

\subsection{Vocabulaire}

\paragraph{Arête contrôlée} Une arête $k$ qui est intégralement occupée
par des troupes d'un unique joueur, qui contrôle les deux nœuds $g_k$
et $d_k$. Ces troupes représentent donc un volume $\gamma_k
\lambda_k$. On note $A^1_q(t)$ l'ensemble des arêtes contrôlées par le
joueur $q$ à l'instant $t$. Par définition, $A^1_q(t) \cap A^1_{\bar
  q}(t) = \emptyset$.

\paragraph{Arête rattachée} Une arête $k$ qui est partiellement occupée
par des troupes. Ces troupes peuvent appartenir à un ou aux deux
joueurs, et peuvent remplir l'arête à partir de $g_k$, de $d_k$, ou
des deux nœuds en même temps. On notera en particulier qu'une arête
non contrôlée sur laquelle un même joueur injecte des troupes depuis
$g_k$ et $d_k$ est une arête rattachée. On note :
\begin{itemize}
  \item $u_{k,q}(t)$ le ou les nœuds par lequel les troupes du joueur
    $q$ peuvent accèder à l'arête rattachée $k$ à l'instant $t$ ;
    $u_{k,q}(t)$ est à valeurs dans $\{\{g_k\}, \{d_k\}, \{g_k,
    d_k\}\}$ ;

  \item $A^2_q(t)$ l'ensemble des arêtes rattachées à $q$ à l'instant
    $t$. Par définition, $A^1_q(t) \cap A^2_q(t) = \emptyset$, mais on
    peut avoir $A^2_q(t) \cap A^2_{\bar q}(t) \neq \emptyset$.
\end{itemize}
Pour toute arête $k \in A^2_q(t) \cup A^2_{\bar q}(t)$, on note :
\begin{itemize}
  \item $x^g_k(t)$ l'abscisse sur l'arête atteinte à l'instant $t$ par
    les troupes arrivant de $g_k$ ;

  \item $x^d_k(t)$ l'abscisse sur l'arête atteinte à l'instant $t$ par
    les troupes arrivant de $d_k$.
\end{itemize}
$x^g_k(t)$ et $x^d_k(t)$ obéissent par ailleurs à la relation :
\begin{equation}
  0 \leq x^g_k(t) < x^d_k(t) \leq \lambda_k
\end{equation}
en considérant par convention que l'abscisse d'ordonnée 0 se trouve au
nœud $g_k$.

\paragraph{Arête disputée} Une arête $k$ qui est intégralement occupée
par des troupes, mais dont les extrêmités gauche et droite sont
occupées par des joueurs différents. On note :
\begin{itemize}
  \item $v_{k,q}(t)$ le nœud par lequel les troupes du joueur $q$
    peuvent accèder à l'arête disputée $k$ à l'instant $t$ ;
    $v_{k,q}(t)$ est à valeurs dans $\{g_k, d_k\}$, et
    $v_{k,q}(t) \cup v_{k,{\bar q}}(t) = \{g_k, d_k\}$ ;

  \item $B(t)$ l'ensemble des arêtes disputées à l'instant $t$. Par
    définition, $B(t) \cap A^1_q(t) = B(t) \cap A^2_q(t) = \emptyset$
    pour tout $q$.
\end{itemize}
Pour toute arête $k \in B(t)$, on note $y_k(t) \in [0 ; \lambda_k]$
l'abscisse à laquelle les troupes de $q$ et $\bar q$ sont en contact à
l'instant $t$.

\paragraph{Arête neutre} Une arête qui ne contient aucune
troupe.

\paragraph{Nœud dépendant} Un nœud relié à une portion d'arête contrôlée
par $q$, rattachée à $q$, ou disputée. Dit autrement, si $i$ est un
nœud dépendant de $q$, il existe $k \in P_i$ tel que :
\begin{itemize}
  \item $k \in A^1_q(t)$ ;

  \item ou $k \in A^2_q(t)$ et $i \in u_{k,q}(t)$ ;

  \item ou $k \in B_q(t)$ et $i = v_{k,q}(t)$.
\end{itemize}
On note $N_q(t)$ l'ensemble des nœuds dépendants de $q$ à l'instant
$t$.

\paragraph{Nœud disputé} Un nœud dépendant à la fois de $q$ et ${\bar q}$.
On note $N^\star(t)$ l'ensemble des nœuds disputés à l'instant $t$ :
$N^\star(t) = N_q(t) \cap N_{\bar q}(t)$.

\paragraph{Ordre} La quantité instantanée de troupes du joueur $q$ qui
quittent l'arête $k$ à l'instant $t$ pour transiter par le nœud
$i$. Pour tout $t$, pour tout $i \in N_q(t)$, pour tout $k \in P_i$,
on note $a_{q,i,k}(t)$ l'ordre émis par le joueur $q$ pour cette
combinaison de nœud et d'arête. Par convention, $a_{q,i,k}(t) > 0$
correspond à un mouvement de troupes depuis l'arête vers le nœud,
alors que $a_{q,i,k}(t) < 0$ correspond à un mouvement de troupes
depuis le nœud vers l'arête. Les ordres sont soumis à des contraintes
(cf.~\ref{déroulé:actions}).

\paragraph{Nœud générateur} Un centre $i \in C_q$ qui génère effectivement
des troupes pour le joueur $q$. Pour cela, le nœud ne doit jamais
avoir été contrôlé par l'ennemi (cf. ci-après). Afin de distinguer un
centre qui ne génère plus de troupes pour le joueur $q$ d'un nœud
générateur, on introduit la notation $\delta_{q,i}(t) \in \{0, 1\}$,
avec la relation :
\begin{equation}
  \label{eq:noeud-generateur}
  \delta_{q,i}(t) = 1 \Leftrightarrow \forall s \leq t, \alpha_{q,i} + \sum_{k \in P_i} \max \left( a_{q,i,k}(s), 0 \right) > \alpha_{{\bar q},i} + \sum_{k \in P_i} \max \left( a_{{\bar q},i,k}(s), 0 \right).
\end{equation}
La quantité de troupes produites à l'instant $t$ par $i \in C_q$ est
alors notée $\alpha_{q,i} \delta_{q,i}(t)$. On note ${\tilde C}_q(t)
\subset C_q$ l'ensemble des nœuds générateurs de $q$ à l'instant $t$.

\paragraph{Nœud contrôlé} Un nœud disputé sur lequel le joueur injecte
plus de troupes que son adversaire à l'instant $t$. Ainsi, le nœud $i
\in N^\star(t)$ est contrôlé par le joueur $q$ à l'instant $t$ si et
seulement si :
\begin{equation}
  \label{eq:noeud-controle}
  \alpha_{q,i} \delta_{q,i}(t) + \sum_{k \in P_i} \max \left( a_{q,i,k}(t), 0 \right) > \alpha_{{\bar q},i} \delta_{{\bar q},i}(t) + \sum_{k \in P_i} \max \left( a_{{\bar q},i,k}(t), 0 \right).
\end{equation}
On note $N'_q(t)$ l'ensemble des nœuds contrôlés par $q$. On notera
qu'on a par ailleurs $N'_{\bar q}(t) = N^\star(t) \backslash N'_q(t)$,
c'est-à-dire l'ensemble des nœuds disputés non contrôlés par $q$.

\paragraph{Nœud ravitaillé} Un nœud $i$ dépendant de $q$ qui est relié à un
nœud générateur de $q$ par un chemin connexe constitué uniquement
d'arêtes contrôlées par $q$ et de nœuds contrôlés par $q$ ou non disputés.

\subsection{Actions des joueurs}
\label{déroulé:actions}

Le joueur $q$ influe sur le plateau de jeu via les ordres
$(a_{q,i,k})$. Ces ordres déterminent les mouvements de leurs troupes,
et doivent respecter à tout instant un ensemble de contraintes.

\paragraph{Nœuds et arêtes non-concernés} Les ordres sur les nœuds non
dépendants sont nécessairement nuls :
\begin{equation}
  \forall q, \forall i \notin N_q(t), \forall k \in P_i, a_{q,i,k}(t) = 0.
\end{equation}

\paragraph{{\'E}quilibre des nœuds} Un nœud ne peut pas
héberger de troupes ; si les soutirages sur des arêtes et les troupes
détruites dans les affrontements ne compensent pas la génération et
les injections de troupes, les troupes surnuméraires sont
perdues. Pour les nœuds disputés contrôlés par $q$, on doit avoir :
\begin{equation}
  \label{eq:equilibre-noeud:disputé:possesseur}
  \forall i \in N'_q(t), \alpha_{q,i} \delta_{q,i}(t) + \sum_{k \in P_i} a_{q,i,k}(t) - \sum_{k \in P_i} a_{{\bar q},i,k}(t) \geq 0.
\end{equation}
Pour les nœuds disputés contrôlés par $\bar q$, on doit avoir pour $q$
:
\begin{equation}
  \forall i \in N'_{\bar q}(t), \forall k \in P_i, a_{q,i,k}(t) \geq 0.
\end{equation}
Pour les nœuds non disputés, on doit avoir :
\begin{equation}
  \forall i \in N_q(t) \backslash N'_q(t), \alpha_{q,i} \delta_{q,i}(t) + \sum_{k \in P_i} a_{q,i,k}(t) \geq 0.
\end{equation}

\paragraph{Quantité maximale d'injection~/~soutirage} Les ordres doivent
respecter la quantité maximale de troupes qu'il est possible
d'injecter ou de soutirer sur l'arête :
\begin{equation}
    \forall i \in N_q(t), \forall k \in P_i, |a_{q,i,k}(t)| \leq \phi_k.
\end{equation}

\paragraph{{\'E}quilibre des arêtes contrôlées} La quantité de troupes
sur une arête contrôlée doit rester constante ; l'injection a une
extrêmité doit donc être exactement compensée par un soutirage à
l'autre extrêmité, et réciproquement :
\begin{equation}
    \forall k \in A^1_q(t), a_{q,g_k,k}(t) = -a_{q,d_k,k}(t).
\end{equation}

\paragraph{Sens de poussée} Les mouvements de troupe ne peuvent aboutir
qu'à une poussée sur les arêtes rattachées à $q$ ou disputées,
c'est-à-dire :
\begin{equation}
  \label{eq:sens-poussée:rattachée}
    \forall k \in A^2_q(t), \forall i \in u_{k,q}(t), a_{q,i,k}(t) \leq 0
\end{equation}
et :
\begin{equation}
  \label{eq:sens-poussée:disputée}
    \forall k \in B(t), a_{q,v_{k,q}(t),k}(t) \leq 0.
\end{equation}

\subsection{{\'E}vénements}

On appelle \og événement \fg{} une situation où le jeu est mis en
pause afin que les joueurs puissent modifier les ordres
$(a_{q,i,k})$. Les événements correspondent aux situations suivantes :
\begin{enumerate}
  \item\label{déroulé:événements:contrôle-arête} une arête rattachée à
    $q$ devient une arête contrôlée par $q$ ;

  \item\label{déroulé:événements:dispute-arête} une arête rattachée à
    $q$ et $\bar q$ devient une arête disputée ;

  \item\label{déroulé:événements:changement-ordre} le joueur $q$
    souhaite modifier ses ordres $(a_{q,i,k})$.
\end{enumerate}
En particulier, l'événement~\ref{déroulé:événements:dispute-arête}
couvre les situations où l'évolution des positions sur le plateau
amène à devoir modifier les ordres afin de continuer à respecter les
contraintes de la sous-section~\ref{déroulé:actions}.

Au moment d'un événement, chaque joueur $q$ peut mettre à jour
l'ensemble de ses ordres $(a_{q,i,k})$, dans le respect des
contraintes de~\ref{déroulé:actions}. Ces modifications sont faites en
parallèle : $q$ modifie ses ordres sans avoir connaissance des
modifications que fera $\bar q$, et réciproquement. Une fois que les
deux jours ont fini de modifier leurs ordres, la partie reprend.

Suite à un événement, un délai de neutralisation est mis en œuvre,
pendant lequel les joueurs ne peuvent pas déclencher un événement du
type~\ref{déroulé:événements:changement-ordre} : cela permet d'éviter
que la partie soit bloquée par des aller-retours incessants pour
contrer les choix de l'adversaire.

\subsection{{\'E}quations d'évolution}

L'évolution de la position des troupes sur une arête dépend de la
quantité de troupes injectée par les joueurs et, éventuellement, de la
présence de l'adversaire\footnote{Pour la suite des équations, on
rappelle que les $(a_{q,i,k})$ sont en convention \og nœud \fg{} et
sont forcèment négatifs ou nuls pour les arêtes rattachées ou
disputées (cf.~\eqref{eq:sens-poussée:rattachée} et
\eqref{eq:sens-poussée:disputée}), ce qui signifie qu'il est
uniquement possible d'injecter des troupes sur l'arête. En outre, sur
une arête $k$, on considère toujours que l'abscisse 0 se trouve au
nœud $g_k$.}.

\paragraph{Cas d'une arête rattachée} Pour les arêtes rattachées,
l'évolution des positions est donnée par :
\begin{equation}
  \forall k \in A^2_q(t) \cup A^2_{\bar q}(t), dx^g_k(t) = -\frac{a^g_k(t)}{\gamma_k} dt \text{~et~} dx^d_k(t) = +\frac{a^d_k(t)}{\gamma_k} dt
\end{equation}
en écrivant par abus de notation $a^g_k(t) = a_{q,g_k,k}(t)$ si $q$
contrôle le nœud $k$, et $a^g_k(t) = 0$ si le nœud $g_k$ n'est
contrôlé par aucun joueur (idem pour $a^d_k(t)$). Sous l'hypothèse que
$a^g_k$ et $a^d_k$ sont des constantes, et toutes choses égales par
ailleurs, on a donc :
\begin{equation}
\begin{cases}
  x^g_k(t) = x^g_k(0)  -\frac{a^g_k}{\gamma_k} t \\
  x^d_k(t) = x^d_k(0) + \frac{a^d_k}{\gamma_k} t.
\end{cases}
\end{equation}
Si $a^g_k + a^d_k < 0$, l'arête change alors d'état à l'instant $t$
tel que :
\begin{equation}
  t = \frac{\gamma_k \left( x^g_k(0) - x^d_k(0) \right)}{a^g_k + a^d_k}.
\end{equation}
Cela correspond à l'une des trois configurations suivantes :
\begin{itemize}
  \item les segments gauche et droit se rencontrent ;

  \item le segment droit est vide et le segment gauche remplit
    intégralement l'arête ;

  \item symétriquement, le segmet gauche est vide et le segmet droit
    remplit intégralement l'arête.
\end{itemize}
Dans tous les cas, cela engendre un événement de type
\ref{déroulé:événements:contrôle-arête} ou
\ref{déroulé:événements:dispute-arête}, en fonction du contrôle des
nœuds $g_k$ et $d_k$.

\paragraph{Cas d'une arête disputée} Pour les arêtes disputées, l'évolution
des positions est donnée par :
\begin{equation}
  \forall k \in B(t), dy_k(t) = \frac{a_{{\bar q},d_k,k} - a_{q,g_k,k}}{2 \gamma_k} dt
\end{equation}
en notant $q$ le joueur qui contrôle le nœud $g_k$ et $\bar q$ son
adversaire. Sous l'hypothèse que $a_{{\bar q},d_k,k}$ et $a_{q,g_k,k}$
sont des constantes, et toutes choses égales par ailleurs, on a donc :
\begin{equation}
  y_k(t) = y_k(0) + \frac{a_{{\bar q},d_k,k} - a_{q,g_k,k}}{2 \gamma_k} t
\end{equation}
Si $a_{{\bar q},d_k,k} - a_{q,g_k,k} > 0$, le joueur $q$ progresse sur
l'arête au détriment de $\bar q$ et l'occupe intégralement à l'instant
$t$ tel que :
\begin{equation}
  t = \frac{2 \gamma_k \left( \lambda_k - y_k(0) \right)}{a_{{\bar q},d_k,k} - a_{q,g_k,k}}
\end{equation}
Inversement, si $a_{{\bar q},d_k,k} - a_{q,g_k,k} < 0$, c'est $\bar q$
qui progresse au détriment de $q$ et occupe intégralement l'arête à
l'instant $t$ tel que :
\begin{equation}
  t = \frac{-2 \gamma_k y_k(0)}{a_{{\bar q},d_k,k} - a_{q,g_k,k}}
\end{equation}
Si le fait qu'un joueur atteigne le nœud adverse change le contrôle de
ce nœud, cela correspond à un événement de
type~\ref{déroulé:événements:contrôle-arête}.

\subsection{Perte de contrôle d'un nœud}
\label{déroulé:perte-nœud}

Le contrôle d'un nœud change dans le cadre d'un événement de type
\ref{déroulé:événements:contrôle-arête} ou
\ref{déroulé:événements:changement-ordre}. Cela correspond à l'instant
où l'inégalité \eqref{eq:noeud-controle} cesse d'être vraie. Le
changement de contrôle d'un nœud a plusieurs effets, qui sont résolus
immédiatement, avant que les joueurs puissent mettre à jour leurs
ordres.

En premier lieu, si $i$ était un nœud générateur pour le joueur $q$,
il cesse de l'être : pour tout $s \geq t$, on a désormais
$\delta_{q,i}(s) = 0$, en application de l'équation
\eqref{eq:noeud-generateur}.

D'autre part, la perte de contrôle de ce nœud est susceptible de
modifier l'ensemble des nœuds ravitaillés par $q$. Les troupes de $q$
présentes sur toutes les arêtes contrôlées par $q$, rattachées à $q$
ou disputées qui ne sont plus reliées à un nœud ravitaillé par $q$
sont immédiatement supprimées de la carte.

\subsection{Fin du jeu}

La partie se termine lorsque le joueur $q$ n'a plus aucun nœud
générateur. Par construction (cf.~\ref{déroulé:perte-nœud}), il n'a
alors plus de troupes sur le plateau. Ce joueur perd la partie et
$\bar q$ est vainqueur.

\printbibliography

\end{document}
