\documentclass[a4paper,11pt,headings=normal]{scrartcl}

\usepackage[utf8]{inputenc}
\usepackage[T1]{fontenc}
\usepackage[french]{babel}
\usepackage{biblatex}
\usepackage{amsthm}
\usepackage{newtx, helvet, courier} % Fontes
\usepackage{booktabs, multirow, hyperref, xcolor}
\usepackage{enumitem}
\usepackage{tikz}
\usetikzlibrary{calc}
\frenchsetup{AutoSpaceFootnotes=false}

\newtheoremstyle{mythm}% name
{0.5\baselineskip plus 0.2\baselineskip minus 0.2\baselineskip}% Space above
{0.5\baselineskip plus 0.2\baselineskip minus 0.2\baselineskip}% Space below
{\itshape}% Body font
{\parindent}% Indent amount
{}% Theorem head font
{.}% Punctuation after theorem head
{.5em}% Space after theorem head
{\thmname{{\scshape#1}}\thmnumber{ #2}\thmnote{~---~#3}}% Theorem head spec (can be left empty, meaning ‘normal’)
\theoremstyle{mythm}
\newtheorem{cas}{Cas}

\definecolor{rouge}{RGB}{191, 35, 41}

\renewcommand{\sectionformat}{\thesection\autodot\enskip}
\renewcommand{\subsectionformat}{\thesubsection\autodot\enskip}
\RedeclareSectionCommand[indent=\the\parindent]{subsection}
\setkomafont{disposition}{}
\setkomafont{section}{\bfseries\large}
\setkomafont{subsection}{\itshape\normalsize}
\setkomafont{paragraph}{\itshape}

\setlist[1]{noitemsep,topsep=0mm,left=\parindent}
\setlist[2]{noitemsep,topsep=0mm,left=0pt}
\setlist[enumerate,1]{label=(\alph*)}

\urlstyle{same}

\addbibresource{citations.bib}

\hypersetup{
    colorlinks=true,
    linkcolor=black,
    citecolor=black,
    urlcolor=blue,
    pdftitle={Mario - Création d'un graphe admissible},
}

\begin{document}

\begin{flushleft}
{\small \sffamily \url{https://github.com/barsanges/mario}}\vspace{0.8em}

{\raggedleft \bfseries\LARGE Génération d'un graphe
  admissible}\vspace{0.8em}

{\itshape Version \input{VERSION}}
\end{flushleft}

\section{Introduction}

Ce document appartient au corpus des spécifications fonctionnelles de
\emph{Mario}, dont le document principal est
\cite{MarioReglesCourant}, et définit une manière de générer un
plateau de jeu admissible pour \emph{Mario}.

\section{Algorithme}
\label{algorithme}

L'algorithme présenté ci-dessous génère à la fois un graphe et des
coordonnées pour représenter les nœuds (et partant le graphe) dans le
plan. Dans la suite, toutes les distances s'entendent en norme $L_2$.

\subsection{Paramètres}
\label{algorithme:parametres}

L'utilisateur doit indiquer les paramètres suivants :
\begin{itemize}
  \item $n$ le nombre de nœuds à générer ;

  \item $0 < \varepsilon < 0.1$ la distance minimale entre deux nœuds
    ;

  \item $1 < k \leq n$ le nombre d'arêtes par nœud (i.e. : pour un
    nœud donné, $k$ arêtes aboutissent à ce nœud) ;

  \item $u \leq v$ les nombres minimal et maximal de centres par camp
    ;

  \item $\omega \in \mathbb{R}^+$ la quantité totale de troupes
    générées par chaque camp au début de la partie.
\end{itemize}

\subsection{Génération des nœuds}
\label{algorithme:noeuds}

On génère $n$ coordonnées distinctes à valeurs dans $[0 ; 1]\times[0 ;
  1]$ par tirage aléatoire suivant une loi uniforme.

Pour chaque paire de nœuds distants l'un de l'autre de moins de
$\varepsilon$, on élague le nœud de la paire le plus proche du point
de coordonnées $(0.5, 0.5)$. On traite les paires de nœuds en fonction
de la distance entre le barycentre des deux nœuds et le point de
coordonnées $(0.5, 0.5)$.

{\`A} la suite de cette étape, il est possible qu'il y ait moins de
$n$ nœuds.

\subsection{Génération des arêtes}
\label{algorithme:aretes}

Pour chaque nœud $i$ généré à l'étape précédente, on crée une arête
entre ce nœud et chacun de ses $k$ plus proches voisins.

On trie alors l'ensemble des arêtes obtenues par longueur croissante
(chaque arête est un segment). On parcourt l'ensemble trié et on
supprime l'arête si elle croise une autre arête. En notant
$((x^A_1,y^A_1), (x^A_2,y^A_2))$ les coordonnées des extrêmités de
l'arête $A$ (avec par convention $x^A_1 \leq x^A_2$), celle-ci croise
l'arête $B$ si et seulement si :
\begin{equation}
  \max \left( x^A_1, x^B_1 \right) \leq \frac{b^A - b^B}{a^A - a^B} \leq \min \left( x^A_2, x^B_2 \right)
\end{equation}
avec :
\begin{equation}
  a^A = \frac{y^A_1 - y^A_2}{x^A_1 - x^A_2}, \text{~resp.~}a^B
\end{equation}
\begin{equation}
  b^A = \frac{y^A_1 x^A_2 - y^A_2 x^A_1}{x^A_2 - x^A_1}, \text{~resp.~}b^B.
\end{equation}
On notera que, à l'issue de cette phase d'élagage, certains nœuds
peuvent avoir moins de $k$ arêtes.

À l'issue de cette phase, on vérifie que le graphe est connexe. S'il
ne l'est pas, la génération s'arrête.

\subsection{Détermination des centres et de leur capacité}
\label{algorithme:centres}

L'utilisateur indique une fourchette $\llbracket u ; v \rrbracket$ du
nombre de centres par camp. Pour le camp $q$ (resp. $\bar q$), on tire
alors un nombre de centres $w \in \llbracket u ; v \rrbracket$, qu'on
choisit de manière aléatoire parmi les centres identifiés à
l'étape~\ref{algorithme:noeuds} situé au-dessous (resp. au-dessus) de
la droite $y = 1 - x$. Si le nombre de centres éligibles est inférieur
à $v$, la génération s'arrête.

Pour chaque camp, on détermine la capacité des $w$ centres retenus en
appliquant l'algorithme suivant :
\begin{enumerate}
  \item {\`A} l'étape $1 \leq k \leq w$, on choisit aléatoirement un
    nœud parmi les $w + 1 - k$ nœuds non encore traités.

  \item On détermine aléatoirement la capacité $\alpha_k$ de ce nœud
    en la tirant dans l'intervalle $[ (\omega - \sum_{i=1}^{k-1}
      \alpha_i) / (w + 2 - k) ; 2 (\omega - \sum_{i=1}^{k-1} \alpha_i)
      / (w + 2 - k)]$, sauf à l'étape $w$ où la capacité du nœud est
    obligatoirement $\omega - \sum_{i=1}^{w-1} \alpha_i$.
\end{enumerate}

\subsection{Détermination des capacités des arêtes}
\label{algorithme:capacites}

Pour déterminer la capacité $\gamma$ des arêtes, on applique
l'algorithme suivant :
\begin{enumerate}
  \item\label{algorithme:capacites:etapes:score} On accorde à chaque
    nœud dont toutes les arêtes non pas été traitées un score
    correspondant à la somme de sa capacité s'il s'agit d'un centre
    (indépendamment du camp auquel il appartient en début de partie)
    et de la somme des capacités des arêtes qui sont reliées à ce nœud
    et qui ont déjà été traitées.

  \item On traite le nœud dont le score est le plus élevé.

  \item\label{algorithme:capacites:etapes:capacites} On définit alors
    la capacité $\gamma_k$ de chacune des arêtes reliées à ce nœud et
    non encore traitées selon la formule suivante :
    \begin{equation}
      \gamma_k = \frac{\theta \lambda_k}{\sum_{i \in U} \lambda_i}
    \end{equation}
    où $\theta$ est le score du nœud calculé à
    l'étape~\ref{algorithme:capacites:etapes:score}, $U$ est
    l'ensemble des arêtes reliées au nœud et non traitées avant le
    début de l'étape~\ref{algorithme:capacites:etapes:capacites}, et
    $\lambda_k$ est la longueur de l'arête $k \in U$.
\end{enumerate}
On réitère ces opérations jusqu'à ce que tous les nœuds (et donc
toutes les arêtes) aient été traités.

\printbibliography

\end{document}
