\documentclass[a4paper,11pt,headings=normal]{scrartcl}

\usepackage[utf8]{inputenc}
\usepackage[T1]{fontenc}
\usepackage[french]{babel}
\usepackage{biblatex}
\usepackage{amsthm}
\usepackage{newtx, helvet, courier} % Fontes
\usepackage{booktabs, multirow, hyperref, xcolor}
\usepackage{enumitem}
\usepackage{tikz}
\usetikzlibrary{calc}
\frenchsetup{AutoSpaceFootnotes=false}

\newtheoremstyle{mythm}% name
{0.5\baselineskip plus 0.2\baselineskip minus 0.2\baselineskip}% Space above
{0.5\baselineskip plus 0.2\baselineskip minus 0.2\baselineskip}% Space below
{\itshape}% Body font
{\parindent}% Indent amount
{}% Theorem head font
{.}% Punctuation after theorem head
{.5em}% Space after theorem head
{\thmname{{\scshape#1}}\thmnumber{ #2}\thmnote{~---~#3}}% Theorem head spec (can be left empty, meaning ‘normal’)
\theoremstyle{mythm}
\newtheorem{cas}{Cas}

\definecolor{rouge}{RGB}{191, 35, 41}

\renewcommand{\sectionformat}{\thesection\autodot\enskip}
\renewcommand{\subsectionformat}{\thesubsection\autodot\enskip}
\RedeclareSectionCommand[indent=\the\parindent]{subsection}
\setkomafont{disposition}{}
\setkomafont{section}{\bfseries\large}
\setkomafont{subsection}{\itshape\normalsize}
\setkomafont{paragraph}{\itshape}

\setlist[1]{noitemsep,topsep=0mm,left=\parindent}
\setlist[2]{noitemsep,topsep=0mm,left=0pt}
\setlist[enumerate,1]{label=(\alph*)}

\urlstyle{same}

\addbibresource{citations.bib}

\hypersetup{
    colorlinks=true,
    linkcolor=black,
    citecolor=black,
    urlcolor=blue,
    pdftitle={Mario - Intelligence artificielle},
}

\begin{document}

\begin{flushleft}
{\small \sffamily \url{https://github.com/barsanges/mario}}\vspace{0.8em}

{\raggedleft \bfseries\LARGE Intelligence artificielle}\vspace{0.8em}

{\itshape Version \input{VERSION}}
\end{flushleft}

\section{Introduction}

Ce document appartient au corpus des spécifications fonctionnelles de
\emph{Mario}, dont le document principal est
\cite{MarioReglesCourant}, et définit la manière dont le joueur
ordinateur $q$ définit ses ordres $(a_{q,i,k})$. La
section~\ref{temporalite} décrit d'abord à quels moments le joueur
recalcule ses ordres, puis la section~\ref{decisions} décrit comment
il les recalcule.

Les notations utilisées dans ce document reprennent celles de
\cite{MarioReglesCourant}.

\section{Instants de prise de décision}
\label{temporalite}

Le joueur ordinateur $q$ recalcule ses ordres dans trois situations
qui correspondent aux trois événements de la section 3.3 de
\cite{MarioReglesCourant} :
\begin{enumerate}
  \item une arête rattachée à un joueur devient une arête contrôlée
    par ce même joueur ;

  \item une arête rattachée à $q$ et $\bar q$ devient une arête
    disputée ;

  \item à la fin du délai de neutralisation suivant l'événement (c) de
    \cite{MarioReglesCourant}, lorsque celui-ci a été déclenché par
    $\bar q$.
\end{enumerate}

\section{Prise de décision}
\label{decisions}

L'algorithme mis en œuvre consiste à déterminer un ensemble d'ordres
$(a_{q,i,k})$ admissibles, au sens où ils respectent l'ensemble des
contraintes de la section~3.2 de \cite{MarioReglesCourant}, et à faire
en sorte que ces ordres maximisent une fonction objectif. Le
paragraphe~\ref{decisions:poids-arete} décrit d'abord comment un poids
est associé à chaque arête en début de partie, puis le
paragraphe~\ref{decisions:fonction-objectif} présente la fonction
objectif que l'on cherche à maximiser pour une prise de décision à
l'instant $t$.

\subsection{Poids associés à chaque arête}
\label{decisions:poids-arete}

En début de partie, un poids $\xi^q_k$ est associée à chaque arête
$k \in \llbracket 1 ; p \rrbracket$ en fonction de sa position par
rapport aux centres de $q$. Pour déterminer les $(\xi^q_k)$, on
applique l'algorithme suivant :
\begin{enumerate}
  \item\label{decisions:poids-arete:etapes:score-noeud} On accorde à
    chaque nœud dont toutes les arêtes non pas été traitées un score
    correspondant à la somme de sa capacité s'il s'agit d'un centre de
    $q$, et de la somme des scores des arêtes qui sont reliées à ce
    nœud et qui ont déjà été traitées.

  \item On traite le nœud dont le score est le plus élevé.

  \item\label{decisions:poids-arete:etapes:score-arete} On définit
    alors le score $\xi^q_k$ de chacune des arêtes reliées à ce
    nœud et non encore traitées selon la formule suivante :
    \begin{equation}
      \xi^q_k = \frac{\theta \gamma_k \lambda_k}{\sum_{i \in U}
        \gamma_i \lambda_i}
    \end{equation}
    où $\theta$ est le score du nœud calculé à
    l'étape~\ref{decisions:poids-arete:etapes:score-noeud} et $U$ est
    l'ensemble des arêtes reliées au nœud et non traitées avant le
    début de l'étape~\ref{decisions:poids-arete:etapes:score-arete}.
\end{enumerate}
On réitère ces opérations jusqu'à ce que tous les nœuds (et donc
toutes les arêtes) aient été traités. On remarquera que cet algorithme
est assez similaire à celui du paragraphe~2.5 de
\cite{MarioGrapheCourant}.

On détermine de la même manière un poids $\xi^{\bar q}_k$ en
fonction de la position de l'arête $k$ par rapport aux centres de
$\bar q$.

Le poids final de l'arête $k$, $\omega_k$, est alors simplement la
somme de $\xi^q_k$ et $\xi^{\bar q}_k$.

\subsection{Fonction objectif}
\label{decisions:fonction-objectif}

{\`A} l'instant $t$, le joueur ordinateur $q$ détermine des ordres
$(a_{q,i,k})$ qui résolvent le problème suivant :
\begin{equation}
  \begin{aligned}
    \max_{(a_{q,i,k})} \quad & -\sum_{k \in A^2_q(t)} \sum_{i \in u_{k,q}(t)} a_{q,i,k}(t) \left( 1 - \sum_{j \in u_{k,{\bar q}}(t)} \left( \frac{a_{{\bar q},j,k}(t)}{\phi_k} \right)^- \right) \omega_k \\
    & \quad - \sum_{k \in B(t)} a_{q,\nu_{k,q}(t),k}(t)  \left( 1 - \left( \frac{a_{{\bar q},\nu_{k,{\bar q}}(t),k}(t)}{\phi_k} \right)^- \right) \omega_k \\
    \text{s.c.} \quad & \forall i \notin N_q(t), \forall k \in P_i, a_{q,i,k}(t) = 0 \\
    & \forall i \in N'_q(t), \alpha_{q,i} \delta_{q,i}(t) + \sum_{k \in P_i} a_{q,i,k}(t) - \sum_{k \in P_i} a_{{\bar q},i,k}(t) \geq 0 \\
    & \forall i \in N'_{\bar q}(t), \forall k \in P_i, a_{q,i,k}(t) \geq 0 \\
    & \forall i \in N_q(t) \backslash N'_q(t), \alpha_{q,i} \delta_{q,i}(t) + \sum_{k \in P_i} a_{q,i,k}(t) \geq 0 \\
    & \forall i \in N_q(t), \forall k \in P_i, -\phi_k \leq a_{q,i,k}(t) \leq \phi_k \\
    & \forall k \in A^1_q(t), a_{q,g_k,k}(t) = -a_{q,d_k,k}(t) \\
    & \forall k \in A^2_q(t), \forall i \in u_{k,q}(t), a_{q,i,k}(t) \leq 0 \\
    & \forall k \in B(t), a_{q,v_{k,q}(t),k}(t) \leq 0 \\
  \end{aligned}
\end{equation}
Ce problème est linéaire, il peut donc être résolu via un solveur de
type \emph{Coin}.

Les décisions prises par ce biais visent à occuper les arêtes les
mieux connectées aux centres des deux joueurs, et à s'opposer de
manière frontale au joueur adverse. Toutefois, elles n'intègrent pas
de coordination à long terme (e.g. : \og si je bloque l'adversaire
ici, est-ce que cela pourrait ouvrir une opportunité plus tard dans la
partie ? \fg{}), et ne sont donc sans doute pas très bonnes.

\printbibliography

\end{document}
